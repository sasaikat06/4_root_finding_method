\documentclass[12pt,a4paper]{article}
\usepackage[utf8]{inputenc}
\usepackage[T1]{fontenc}
\usepackage{amsmath}
%\usepackage{titlesec}
\usepackage{amsfonts}
\usepackage{mathptmx}
\usepackage{amssymb}
\usepackage{graphicx}
\usepackage{hyperref}
\usepackage{multicol}
\usepackage{geometry}
\usepackage{fancyhdr}



\begin{document}
	\tableofcontents
	\thispagestyle{empty}
	
	\section{Green's Theorem}
	
	\title{Green's Theorem in Complex Form} \textbf{Green's Theorem in Complex Form}
	
	Green's theorem in complex analysis relates a contour integral around a closed curve to a double integral over the region enclosed by the curve. The complex form of Green's theorem expresses the relationship between a contour integral and partial derivatives of the real and imaginary parts of a complex function. \\
	
	 \textbf{\underline{Theorem:} Green's Theorem in Complex Form:}
	
	\textbf{Statement:}	Let \( f(z) = u(x, y) + iv(x, y) \) be a continuously differentiable complex function, where \( u(x, y) \) and \( v(x, y) \) are the real and imaginary parts of \( f \), respectively. Suppose \( \gamma \) is a positively oriented, simple, closed curve that encloses a region \( R \) in the complex plane, and the function is defined and continuously differentiable on \( R \cup \gamma \). Then, the contour integral of \( f(z) \) around \( \gamma \) can be related to a double integral over the region \( R \) as follows:
	\[
	\oint_{\gamma} f(z) \, dz = 2i \iint_R \left( \frac{\partial v}{\partial x} - \frac{\partial u}{\partial y} \right) \, dx \, dy.
	\]
	
 \textbf{Proof:}
	
	 
	
	Let \( f(z) = u(x, y) + iv(x, y) \), where \( u(x, y) \) and \( v(x, y) \) are the real and imaginary parts of the complex function \( f \), and \( z = x + iy \) is a point in the complex plane. The contour integral of \( f(z) \) around the closed curve \( \gamma \) is:
	\[
	\oint_{\gamma} f(z) \, dz = \oint_{\gamma} (u(x, y) + iv(x, y)) \, (dx + i \, dy).
	\]
	We expand this as:
	\[
	\oint_{\gamma} f(z) \, dz = \oint_{\gamma} \left( u(x, y) \, dx - v(x, y) \, dy \right) + i \oint_{\gamma} \left( u(x, y) \, dy + v(x, y) \, dx \right).
	\]
	This separates the contour integral into its real and imaginary parts.
	

	
	Now, apply the standard *Green's Theorem* (from vector calculus) to these two line integrals. Green's theorem in real form states that for continuously differentiable scalar fields \( P(x, y) \) and \( Q(x, y) \), the following holds:
	\[
	\oint_{\gamma} P \, dx + Q \, dy = \iint_R \left( \frac{\partial Q}{\partial x} - \frac{\partial P}{\partial y} \right) \, dx \, dy.
	\]
	1. \( \oint_{\gamma} \left( u(x, y) \, dx - v(x, y) \, dy \right) \), we set \( P(x, y) = u(x, y) \) and \( Q(x, y) = -v(x, y) \). Applying Green's theorem:
	\[
	\oint_{\gamma} \left( u(x, y) \, dx - v(x, y) \, dy \right) = \iint_R \left( \frac{\partial (-v)}{\partial x} - \frac{\partial u}{\partial y} \right) \, dx \, dy.
	\]
	Simplifying this:
	\[
	\oint_{\gamma} \left( u(x, y) \, dx - v(x, y) \, dy \right) = \iint_R \left( -\frac{\partial v}{\partial x} - \frac{\partial u}{\partial y} \right) \, dx \, dy.
	\]
	
	2.  \( \oint_{\gamma} \left( u(x, y) \, dy + v(x, y) \, dx \right) \), we set \( P(x, y) = v(x, y) \) and \( Q(x, y) = u(x, y) \). Applying Green's theorem:
	\[
	\oint_{\gamma} \left( u(x, y) \, dy + v(x, y) \, dx \right) = \iint_R \left( \frac{\partial u}{\partial x} + \frac{\partial v}{\partial y} \right) \, dx \, dy.
	\]

	
	Now, combining both integrals, the contour integral becomes:
	\[
	\oint_{\gamma} f(z) \, dz = \iint_R \left( -\frac{\partial v}{\partial x} - \frac{\partial u}{\partial y} \right) \, dx \, dy + i \iint_R \left( \frac{\partial u}{\partial x} + \frac{\partial v}{\partial y} \right) \, dx \, dy.
	\]
	
	But notice that the imaginary part of the contour integral simplifies due to the Cauchy-Riemann equations, which say that for a holomorphic function, \( \frac{\partial u}{\partial x} = \frac{\partial v}{\partial y} \) and \( \frac{\partial u}{\partial y} = -\frac{\partial v}{\partial x} \). Therefore, the contour integral simplifies to:
	\[
	\oint_{\gamma} f(z) \, dz = 2i \iint_R \left( \frac{\partial v}{\partial x} - \frac{\partial u}{\partial y} \right) \, dx \, dy.
	\]
	
	 Conclusion:
	
	This is the *complex form of Green's theorem*, which relates the contour integral of a function around a closed curve to the double integral over the enclosed region involving the partial derivatives of the real and imaginary parts of the function. \\ \\ \\ 
	\newpage
	
	
	\section{Grad, Div,Curl, Laplacian}
	\title{Grad,Div,Curl} \textbf{Grad, Div,Curl, Laplacian} \\
In the context of complex analysis, the gradient, divergence, curl, and Laplacian operators are primarily defined for vector fields in real variables, but they can be related to complex functions. Here’s how they can be represented in complex form for two-dimensional functions involving the complex variable \( z = x + iy \), where \( f(z) \) is a complex function.

 1. Gradient ($\bigtriangledown$)
The gradient is a vector operator that describes the rate and direction of change of a scalar field. In terms of a real-valued function \( f(x, y) \), the gradient in two dimensions is given by:

\[
\nabla f = \left( \frac{\partial f}{\partial x}, \frac{\partial f}{\partial y} \right)
\]

In terms of the complex variable \( z \), where \( z = x + iy \), we can write the partial derivatives using the *Wirtinger derivatives*:

\[
\frac{\partial}{\partial z} = \frac{1}{2}\left(\frac{\partial}{\partial x} - i\frac{\partial}{\partial y}\right), \quad \frac{\partial}{\partial \bar{z}} = \frac{1}{2}\left(\frac{\partial}{\partial x} + i\frac{\partial}{\partial y}\right)
\]

Thus, the gradient in complex form can be written as:

\[
\nabla f(z) = 2 \frac{\partial f}{\partial z}
\]

 2. *Divergence ($\bigtriangledown$·)*
The divergence measures the rate of change of a vector field flowing out of a region. In two dimensions, for a vector field \( \mathbf{F}(x, y) = (F_x(x, y), F_y(x, y)) \), the divergence is:

\[
\nabla \cdot \mathbf{F} = \frac{\partial F_x}{\partial x} + \frac{\partial F_y}{\partial y}
\]

For a complex function \( f(z) = u(x, y) + iv(x, y) \), where \( u(x, y) \) and \( v(x, y) \) are the real and imaginary parts, the divergence is related to the real part of the complex derivative:

\[
\nabla \cdot \mathbf{F} = 2 \, \text{Re} \left( \frac{\partial f}{\partial z} \right)
\]

3. *Curl ($\bigtriangledown$×)*
The curl measures the rotation of a vector field. In two dimensions, the curl of a vector field \( \mathbf{F}(x, y) = (F_x, F_y) \) is a scalar and is given by:

\[
\nabla \times \mathbf{F} = \frac{\partial F_y}{\partial x} - \frac{\partial F_x}{\partial y}
\]

For a complex function \( f(z) \), the curl is related to the imaginary part of the complex derivative:

\[
\nabla \times \mathbf{F} = 2 \, \text{Im} \left( \frac{\partial f}{\partial z} \right)
\]
4. *Laplacian ($\bigtriangledown$²)*
The Laplacian operator describes the divergence of the gradient of a scalar field and is used in solving partial differential equations like the heat or wave equation. In two dimensions, it is defined as:

\[
\nabla^2 f = \frac{\partial^2 f}{\partial x^2} + \frac{\partial^2 f}{\partial y^2}
\]

In terms of complex derivatives, the Laplacian of a function can be written as:

\[
\nabla^2 f = 4 \frac{\partial^2 f}{\partial z \, \partial \bar{z}}
\]

This is because the Laplacian is proportional to the mixed partial derivatives with respect to \( z \) and \( \bar{z} \).

Summary
- *Gradient* in complex form: \( \nabla f = 2 \frac{\partial f}{\partial z} \)
- *Divergence*: \( \nabla \cdot \mathbf{F} = 2 \, \text{Re} \left( \frac{\partial f}{\partial z} \right) \)
- *Curl*: \( \nabla \times \mathbf{F} = 2 \, \text{Im} \left( \frac{\partial f}{\partial z} \right) \)
- *Laplacian*: \( \nabla^2 f = 4 \frac{\partial^2 f}{\partial z \, \partial \bar{z}} \)

These forms connect complex functions with the standard operators in vector calculus.

\newpage

\section{L'Hopitals' Rule}

*L'Hopital's Rule* is a method used in calculus to evaluate limits of indeterminate forms, such as \( \frac{0}{0} \) or \( \frac{\infty}{\infty} \). The rule provides a way to simplify such limits by differentiating the numerator and denominator until the limit can be evaluated.

+ *Statement of L'Hopital's Rule:*

Let \( f(x) \) and \( g(x) \) be functions that are differentiable on an open interval \( (a, b) \) containing the point \( c \), except possibly at \( c \) itself. Suppose:

1. The limit \( \lim_{x \to c} \frac{f(x)}{g(x)} \) results in an indeterminate form \( \frac{0}{0} \) or \( \frac{\infty}{\infty} \).
2. The derivatives \( f'(x) \) and \( g'(x) \) exist and \( g'(x) \neq 0 \) for all \( x \) near \( c \).

Then, 

\[
\lim_{x \to c} \frac{f(x)}{g(x)} = \lim_{x \to c} \frac{f'(x)}{g'(x)}
\]

*Important Notes:*
- L'Hopital's Rule can be applied repeatedly if the result remains indeterminate after applying it once.
- The rule only works when the limit gives an indeterminate form like \( \frac{0}{0} \) or \( \frac{\infty}{\infty} \).

*Example:*
Evaluate \( \lim_{x \to 0} \frac{\sin x}{x} \).

- The direct substitution \( \lim_{x \to 0} \frac{\sin x}{x} = \frac{0}{0} \), an indeterminate form.
- Applying L'Hopital's Rule:

\[
\lim_{x \to 0} \frac{\sin x}{x} = \lim_{x \to 0} \frac{\cos x}{1} = \cos(0) = 1
\]

Thus, \( \lim_{x \to 0} \frac{\sin x}{x} = 1 \).

\end{document}